% !TEX root = ../main.tex
\chapter{Conclusions}
\section{Summary of the Thesis}
Assessment of biomechanics in natural environments could empower clinicians to tailor preventative or rehabilitation therapies of knee diseases in ways traditional gait laboratories have not allowed. Compared with grounded equipment, low-cost, light-weight, and small-size wearable IMUs create little discomfort during walking so they can markedly increase the assessment accessibility and thus promote gait training and outdoor running injury prevention. However, new models and algorithms are needed to extract meaningful outcomes from these new modalities and then translate them to out-of-the-lab applications. Although many kinematics-based, physics-based, and data-driven models have been proposed for knee load assessment, the accuracy and robustness of these models are still hindered by sensor placement error, magnetometer dependency, suboptimal sensor configuration, and inapplicability to gait modifications. Also, these models and algorithms were mostly tested on a normal walking and running style, so their validity across different gait modifications required by clinical applications is unknown. To solve these problems, the following research was conducted:

1) This thesis systematically investigated the influence of eight synthesized IMU sensors' position and orientation placement errors on GRF estimation. 1888 different placement errors were tested, including baseline cases when a single sensor was misplaced and extreme cases when all sensors were simultaneously misplaced. It was concluded that orientation errors would more adversely affect accuracy than position errors for both single and multiple simultaneous placement error conditions. Therefore, when researchers and therapists use IMUs for kinetic estimation, they should carefully perform IMU (especially trunk IMU) orientation placement.

2) This thesis proposed a CNN model for VALR estimation using acceleration and angular velocity measured by body-worn IMUs. To make the proposed model applicable to various running conditions, data were collected with combinations of different foot-strike patterns, step rates, running speeds, and footwear. To determine the optimal sensor placement configuration, thirty-one configurations of between one to five IMUs at the trunk, pelvis, thigh, shank, and foot were evaluated. VALR estimation via the CNN model with a single IMU at the shank was highly correlated ($\rho = 0.94$) with force-plate VALR measurements and was substantially higher than previously reported peak tibial acceleration correlations with force-plate VALR measurements from shank-worn accelerometers ($\rho = 0.44 - 0.66$). There was no improvement in model estimation accuracy when including additional data from any combination of the other four IMUs. This work could enable runners to assess impact loading rates more accurately and potentially provide insights into running-related knee injury risk and prevention.

3) This thesis proposed a foot-worn IMU-based FPA estimation algorithm comprised of three key components: orientation estimation, acceleration transformation, and FPA estimation via peak foot deceleration. Without using a magnetometer for heading drift correction, the proposed algorithm can still provide comparable FPA estimation accuracy (MAE = 3.1±1.3 deg) during continuous straight walking and higher validity after walking starts and turns compared with previous magneto-IMU-based algorithms. This work could enable FPA assessment in environments where magnetic field distortion is present due to ferrous metal structures and electrical equipment, or in real-life walking conditions when walking starts, stops, and turns commonly occur.

4) This thesis proposed an IMU-based RNN model for KAM and KFM estimation. Eight IMU sensors were placed on trunk, pelvis, both thighs, shanks, and feet. Features were extracted from IMU data, estimate GRF and lever arm components, and finally compute knee moments. To make the proposed model applicable to various gait modifications, the author collected subject walking data at three different speeds with modifications of step width, FPA, and trunk sway angle. The proposed model was valid across all the tested walking gaits including decreased/increased step width, toe-in/toe-out gaits, and small/large trunk sway angles. The proposed model could enable knee moment assessment in various environments such as clinics, homes, or athletic facilities.

\section{Contributions}
This thesis focused on developing novel models to translate knee load assessment from gait laboratories to real-life walking and running environments. Wearable IMUs were used to model kinematic and kinetic parameters related to knee load. Compared with previous works, this thesis elaborately presented the influence of IMU placement errors, and developed magnetometer-free knee load estimation models with optimally placed sensors. Those models were validated across various gait patterns. The detailed contributions are:

1) This thesis systemically investigated the placement error influences by synthesizing 1888 different errors across eight major body segments. The importance of IMU orientation placement was emphasized in implementations of kinetic estimation. Researchers were suggested to take error-correction measures.

2) This thesis performed experimental testing and statistical analysis to determine the optimal IMU location configuration for assessing knee load. It was concluded that one shank-worn IMU was the optimal configuration for assessing impact loading, and the eight-IMU-based model was not significantly more accurate than the three-IMU-based model for assessing knee moments.
%  (trunk, pelvis, both thighs, both shanks, and both feet) 
% (pelvis and both feet) 

3) This thesis proposed IMU-based models and algorithms that do not rely on magnetometers. One representative example is the magnetometer-free FPA estimation algorithm, which only used accelerometer and gyroscope data to derived FPA. The proposed models can fulfill the need of knee load assessment in environments where magnetic field distortion is present.
% This thesis proposed a novel magnetometer-free FPA estimation algorithm based on a foot-worn IMU. FPA was derived from the peak deceleration of the foot before each heel-strike event. The proposed algorithm fulfills the needs of monitoring FPA 

4) This thesis proposed models that are valid for various gait modifications. Specifically, it was demonstrated that the VALR estimation model achieved high correlations across different strike patterns and step rates, and the KAM and KFM estimation model was applicable for various modified gaits. The proposed models can enable out-of-the-lab gait training studies aiming at running injury prevention or knee moment reduction.

\section{Future Work}
Recent advances in biomechanics have demonstrated that wearable IMUs are promising in out-of-the-lab knee load assessment. To strengthen the proposed IMU-based knee load estimation models and ultimately make them widely accepted by doctors and clinicians, more efforts need to be devoted into:

1) Standardized datasets: 
Data-driven approaches are promising in modeling kinetics that was difficult to be modeled via physics-based approaches. Many recent machine learning breakthroughs in computer science fields are based on the massive amount of data.
However, the number of subjects of open-source gait datasets is commonly less than one hundred, which is significantly smaller than the size of image or audio datasets. 
Human subject data are scarce and valuable because their collection is time-consuming and labor-intensive. Hours are needed to collect data of one single subject and experiences are needed to place the sensors, operation of specialized equipment, and monitoring of the subjects.
Currently, there are many gait datasets collected by different laboratories available online, but differences exist in their subject population, sensor type, or reflective marker locations. Due to these differences, it is difficult to validate the knee load assessment models across datasets. Also, when researchers try to find useful resources for their projects, searching for datasets and merging usable ones would be non-trivial work. Thus, to make the biomechanics society benefit from big data, recommendations should be published to instruct researchers to standardize their experimental protocol design, data processing operations, and data format before making them available online. 

2) Combination of physics-based and data-driven approaches:
Physics-based and data-driven models were categorized into two different kinetic modeling approaches in the state-of-the-art research section of this thesis. Few studies have attempted to combine these two approaches, which might potentially improve the overall performance. Here are some examples. First, domain knowledge from physics-based models can be referred to when extracting features for data-driven models. Second, human musculoskeletal modeling can be used to synthesize gait data according to the biometrics of a specific subject, which can be used as subject-specific training data to improve the accuracy of data-driven models. Third, physics-based models simplify the human body using parameterized formulas. Instead of setting these parameters empirically or using population average, they can be set from collected data using machine learning methods so that they are optimal values with no bias for the target population.

3) Modeling for patients:
Compared with healthy people, the walking gaits of patients with knee osteoarthritis or injuries are significantly different with more variance. The collection of patient data is more difficult because of the inconvenience in traveling, movement troubles, and lack of effective communication. As a result, many of the existing models were validated on cohorts of healthy subjects despite that patients were their target users. Future research should recruit patients at different disease stages and evaluate the generalizability of models. Additionally, in case of poor performance of data-driven models caused by insufficient data, transfer learning can be one potential solution. Many datasets are available online with healthy subject data, so using transfer learning to transform these data can significantly enhance the limited patient data.

4) Model validation outside the gait laboratory:
Most of the existing models were validated inside gait laboratories against the OMC and embedded force plates. However, compared to out-of-the-lab conditions, laboratory environments might affect subject's gait because they cannot walk/run on the floor for a long distance or on a treadmill with self-selected accelerations and speeds. To make patients and clinicians trust these new models, experiments should be performed outside the gait laboratory to demonstrate these models' effectiveness in clinical applications. For example, clinical experiments could be performed to validate the effectiveness of wearable-based gait training in reducing knee pain and slowing down the progress of cartilage degradation. Also, controlled experiments could be performed to determine whether these wearable sensors could reduce the risk of injury during athlete training.





